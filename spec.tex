\documentclass{article}
\usepackage{geometry} 
\usepackage{verbatim}
\geometry{letterpaper} 

\title{Specification for Redwood}
\author{Andy Balholm}

\begin{document}

\maketitle

\begin{abstract}

Redwood is an internet content-filtering program. 
It is designed to replace and improve on DansGuardian 
as the core of the Security Appliance internet filter. 
It adds flexibility and granularity to the filtering by classifying sites 
into multiple categories instead of just ``Allow'' and ``Block.'' 

\end{abstract}

\section{Basic Architecture}

Redwood runs as an HTTP proxy server or as an ICAP server.
It examines each HTTP message to determine if it should be allowed to proceed. 
If so, it passes the message on 
to its final destination. If not, it replaces the message with a customizable page 
stating that the request is not allowed (optionally giving the reason 
and providing a link for filing an overblock request).

Redwood's filtering is based on URLs and also, where applicable, on page content. 
(By default, content filtering is applied to files whose MIME type starts with \verb"text".)

\section{Configuration File}

By default, the main configuration file is located at \verb"/etc/redwood/redwood.conf". 
This path can be changed by using the \verb"-c" command line switch. 
Configuration options may be specified either in the configuration file or as command-line
switches. In the configuration file, they may be specified either as
\verb"key = value" or as \verb"key value". Comments are delimited with \verb"#". 
Values may be enclosed in double quotes, with the usual backslash escapes.
Additional configuration files may be included by using the \verb"include" directive.

An example configuration file:

\begin{verbatim}
# Listen for connections on port 8000.
http-proxy :8000

# Also run as an ICAP server, listening on port 1344.
icap-server :1344

# the template for the block page
blockpage "/etc/redwood/block.html"

# directory of static files to be served by the internal web server
static-files-dir /etc/redwood/static

# directory of CGI scripts to run by the internal web server
cgi-bin /etc/redwood/cgi

# the directory containing the category information
categories /etc/redwood/categories

# the minimum total score from a blocked category needed 
# to block a page
threshold 275

# file configuring the content pruning
content-pruning /etc/redwood/pruning.conf

# file configuring URL query modification
query-changes /etc/redwood/safesearch.conf

# path to the access log
access-log /var/log/redwood/access.log

# Customize which MIME types get filtered.
mime-allow text/css # Don't filter CSS files.
mime-filter application/javascript # Filter JS
mime-filter application/x-javascript
mime-filter application/json
mime-block video/x-ms-wmv # Block Windows Media Player video.
\end{verbatim}

\section{Categories}

The configuration files for Redwood allow the user to establish any number of categories 
corresponding to the types of content that he wishes to block or to allow. 
As each HTTP message is processed, it is assigned a score in each category, 
based on the filter lists that are set up for that category. 
These scores are then used to determine whether the page should be blocked.

Each category is a assigned an action: allow, block, or ignore. 
A page will be blocked if the score for any category listed as \verb"block" 
is higher than the highest score for any category listed as \verb"allow". 
If a category is listed as \verb"ignore", its score does not affect whether a page 
is blocked or not. (This is the default for a category that is not listed.) 
However, a page is not blocked unless the score for the highest \verb"block" category
is greater than a certain configurable threshold. This prevents overblocks 
of pages with almost no textual content.

The categories are stored in a directory whose location is specified in the configuration file. 
Each subdirectory of that directory defines a category (with the same name as the directory). 

Each category's directory contains a file named \verb"category.conf" and any number of rule-list files. 
A category named ``mechanical'' might have a \verb"category.conf" file like the following:

\begin{verbatim}
description: Auto Repair
action: allow
\end{verbatim}

This configuration would mean that the category's user-visible description would be 
``Auto Repair'' rather than ``mechanical,'' and that pages that fall into the category would be allowed. 
The description defaults to the category name, and the action 
defaults to \verb"ignore". Actions can be overriden for specific users by the use of filter groups.
A \verb"category.conf" file may also have the entry \verb"invisible: true"; 
this indicates that when a page is blocked because it belongs to that category,
the response will be an invisible image instead of the usual block page.

The rule-list files define the rules used to calculate the category's score. 
Each rule-list file must have an extension of \verb".list". 
(This rule ensures that files ending in \verb".bak", \verb".orig", etc. are ignored.) 
It is a plain-text file encoded in UTF-8. Comments are delimited with \verb"#". 
Here is an example of a rule-list file that might be in the directory for the 
``mechanical'' category mentioned earlier:

\begin{verbatim}
napaonline.com 200 # Give napaonline.com 200 points for this category.
www.napaonline.com/catalog/ 50 # bonus points for NAPA's catalog

default 150 # The following domains will each get 150 points.
carquest.com
autozone.com

/t[iy]re/ 75 # Any page with tire or tyre in the URL will get 75 points.
/parts/h 50 # A page with parts in the hostname will get 50 points

<grease gun> 25 # 25 points for each occurrence of "grease gun" in the content
<oil filter> 25 100 # 25 points for each occurrence, but no more than 100 total
\end{verbatim}

There are three kinds of filter rules:

\begin{description}

\item[URL matching] A URL matching rule consists of a domain name, optionally followed by a path. 
After it, separated by a space, is the weight---the number of points that get added to this category's score 
for sites that match the rule. 

A rule for a domain will also match subdomains: \verb"napaonline.com" also matches \verb"www.napaonline.com".
A rule with a path will also match longer paths: \verb"www.napaonline.com/catalog" also matches 
\verb"www.napaonline.com/catalog/result.aspx".

If a domain and a subdomain (or a path and a subdirectory) 
are both listed, the subdomain will effectively get the sum of the two weights. For example, if \verb"xerox.com"
were listed with 100 points, and \verb"support.xerox.com" were listed with 50 points, \verb"support.xerox.com" 
would actually get a score of 150 points.

\item[URL regular expressions] A regular expression to match the URL is listed between slashes. 
The points are added to the category score for each page whose URL matches the regular expression. 
The URL is converted to lower case before comparing it to the regular expressions.
The regular expression syntax is that supported by the RE2 library.

A regular expression can be restricted to matching a certain part of the URL by adding a one-character suffix
immediately after the final slash.
A suffix of \verb"h" matches the hostname (e.g. \verb"www.google.com"), 
\verb"d" matches the base domain name (e.g. \verb"google"),
\verb"p" matches the path,
and \verb"q" matches the query.

\item[Content phrases] Unlike the other two kinds of rules, these apply to the content of the page, not the URL. 
Phrases are enclosed between angle brackets. Before testing to see if a phrase matches, 
both the phrase and the page are simplified: capital letters are converted to lowercase, 
all characters that are not letters or digits are replaced by spaces,
and multiple spaces are replaced by single spaces. Then the phrase weight is added to the page's score for the category 
for each time the phrase is found on the page. But if the phrase has a second weight listed, 
no more than that amount will be added no matter how many times the phrase occurs. 
(In the example, if ``oil filter'' occurred more than four times, the additional occurrences wouldn't count.)

\end{description}

There is also a \verb"default" rule. It specifies what weight will be assigned to rules that don't specify a weight. 
It applies to all rules without a specified weight between it and the next \verb"default" rule or the end of the file. 
If there is no \verb"default" rule, the default weight is zero.

Weights must be integers, but they may be negative. Negative weights can be used to offset short, general matches 
with long, more-specific ones, e.g.:

\begin{verbatim}
<grease> 10
<grease paint> -10
\end{verbatim}

If a page is blocked based on its URL (i.e. by URL matching and/or URL regular expressions), 
its content will not be evaluated because the page will not be downloaded.

\section{URL Query Modification}

When processing an HTTP request, Redwood can modify the query parameters in the URL.
The configuration file for these changes is specified with the \verb"query-changes" keyword.
Each line contains a URL-matching or URL-regular-expression rule, 
followed by a query expression.
If the query in the URL already contains parameters with the same names as those specified in the file,
they will be replaced with the new values. Otherwise the new values will be added.

\begin{verbatim}
# Force safe search on several search engines.
/www\.google\.[^/]+/search/ safe=vss
search.lycos.com adv=1&adf=on
search.yahoo.com vm=r
/hotbot/h adf=on
www.metacrawler.com familyfilter=1
\end{verbatim}

\section{Content Pruning}

Between downloading a page and scanning its content for phrases, 
Redwood can perform ``content pruning.'' 
This is scanning the parsed HTML tree for elements matching certain criteria,
and deleting those elements and their children.

Content pruning is controlled by a configuration file.
Each line of the file contains a URL-matching or URL-regular-expression rule
to specify what site or page the pruning applies to, 
and a CSS selector to specify what elements to delete.

\begin{verbatim}
# Craigslist personals and discussion forums
craigslist.org div#ppp, div#forums, option[value=ppp]

# Bing ad sidebar
bing.com div.sb_adsNv2
\end{verbatim}

\section{Block Pages}

When Redwood blocks access to a web page, 
it returns an HTTP response with a status of 404 Forbidden.
Unless the category that caused the page to be blocked is configured as \verb"invisible",
the body of the 404 response will be HTML rendered from a template file.
The template file is specified with the \verb"blockpage" configuration directive.
The following placeholders may be used in the template file, 
to be replaced by the appropriate information when the block page is sent:

\begin{description}

\item[\{\{.URL\}\}] the URL of the page that was blocked
\item[\{\{.Categories\}\}] the names of the categories that caused the page to be blocked
\item[\{\{.User\}\}] the user's IP address or username
\item[\{\{.Group\}\}] the group the user belongs to, or an empty string
\item[\{\{.Tally\}\}] a list of the rules that matched, and how many times each one matched
\item[\{\{.Scores\}\}] a list of categories, and how many points the page scored in each category

\end{description}

The block page is generated using the Go template package; see
\verb"http://golang.org/pkg/text/template"
and
\verb"http://golang.org/pkg/html/template"
for documentation.

There is one custom function defined for the templates to use, \verb"eq", 
which tests its parameters for equality.
For example, \verb[{{if eq .Group "admin"}}You're an administrator!{{end}}[

Since the block page may need to refer to external resources
(such as images, stylesheets, and scripts),
Redwood includes an internal web server.
This web server does not accept connections directly,
but whenever Redwood processes a request with a server address of 203.0.113.1,
it directs the request to the internal server instead of processing it normally.

\section{Test Mode}

If Redwood is run with the \verb"-test" switch, it does not run as an ICAP server. 
Instead, it evaluates the URL given as an argument after the switch.
It prints detailed debugging information about how the URL and its content would be rated
if that page were requested in normal operation: how many times each rule matches, 
what the score is in each category, which categories would block the page, etc.

\section{Log Files}

Redwood has two categories of messages that can be logged:
general diagnostic messages, and a record of ICAP requests processed (called the access log).
The general diagnostic messages are sent to standard error by default,
and may be redirected to a file using normal shell redirection.
The access log is in CSV format, 
and goes to standard output by default.
It can be sent to a file by including the \verb"access-log" directive in Redwood's configuration file.

The access log has the following fields: time, username or IP address, 
action (allow, block, or ignore), URL, HTTP method (GET, PUT, etc.),
HTTP response status (if an HTTP response was being processed), 
content type, content-length, whether the content was modified by Redwood, which rules matched (and how many times), 
the score for each category, and the list of categories that caused the page to be blocked (if it was).
The content length is meaningful only if a phrase scan was performed.

\section{Filter Groups}

Users may be divided into groups, with each group having access to different categories of sites.
A user is assigned to a group with the \verb"group" directive in the configuration file.
For example, \verb"group admin 10.1.10.203" would assign the user at IP address 10.1.10.203
to the group named ``admin.'' 
If a user authenticates with the proxy using a username and password, 
his username would be used instead of his IP address.

The filtering for a filter group is configured with the \verb"block", \verb"allow", and \verb"ignore"
directives. These directives override the \verb"action" line in the \verb"category.conf" file,
but only for users that belong to a specific filter group.
For example, \verb"allow proxies admin" would allow users in the ``admin'' group
to access sites in the ``proxies'' category even if they were normally blocked.

When the \verb"block", \verb"allow", and \verb"ignore" directives are used without a group name,
they apply to all users---except that groups may override them with different actions.

\section{Authentication}

When a client connects to Redwood's HTTP proxy server from outside the local network,
Redwood requires HTTP basic proxy authentication, with a username and password.
The usernames and passwords come from a file that is specified by the \verb"--password-file"
configuration directive. Each line in the file consists of a username,
a space or a tab, and a password.
If no password file is specified, no connections from outside the LAN will be accepted.

\end{document}
